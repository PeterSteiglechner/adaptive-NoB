Dear Theresa Lackner and Organisers of the Workshop, 

Social influence is a key driver of people's opinions on political issues and thus plays a role in driving or hindering social change. While there is extensive modelling literature on the effects of social influence on macro-patterns such as polarisation or consensus, many models have relatively sparse psychologically plausible micro-foundations. One aspect that has been particularly neglected in the modelling literature is that individuals often perceive the world in subjective ways, shaped by their own personal contexts and experiences. I want to contribute to the modelling literature by exploring and illustrating how subjective, often heterogeneous perceptions of different individuals, including interpretations of political events, perceptions of others' opinions, or feelings of which beliefs are coherently related and which not, can drive social dynamics and change the way certain models function (see also the attached abstract, as well as my manuscripts on social identity distortions in perceptions \cite{steiglechnerHowOpinionVariation2025c, steiglechnerSocialIdentityBias2023} and opinion formation under ambiguous messages \cite{steiglechnerNoiseOpinionDynamics2024, steiglechnerMakeNoiseWhy2025}).

Goals like this rely, of course, on fruitful \textit{exchange} with fellow opinion dynamics modellers and experts in the social sciences. I have found that winter/summer schools (as a lecturer and as a participant) are great opportunities to do that: to learn from others, to sharpen one's understanding of the purposes of modelling, to distribute and to hear about novel ways of looking at (social) phenomena, and to share ideas for future directions in the field. One of the key questions, which is also highlighted in the announcement of the Graz winter school/workshop, is the \textbf{link between theoretical modelling work and empirical data}. To me, this challenge fits well into the specific goal to better understand how people perceive seemingly `true/objective data' in subjective and often different ways and how this affects the dynamics. For example, content from political social media posts might carry a stance on a particular debated issue---something that we can measure from a scientist's perspective. But it is unclear whether real people perceive it in a similar way and how biases or other distortion factors affect how individuals evaluate and interpret the content of such a post from their diverse and subjective viewpoints. While many `socio-physics' models are overly simplistic for application to real-world social phenomena, linking models to reality through data-driven approaches is not that straight-forward. I believe that this challenge is best tackled with an \textit{integrated approach} of constraining phenomena/processes through real-world and experimental data, illustrating and exploring `what-if' style questions through modelling, taking lessons from complex systems research, and being bold on formalising social science theories.

The Graz Schumpeter Winter School 2025 and COLIBRI Focus Workshop on `Modelling Opinion Dynamics and Social Change' will surely offer a fresh and inspiring perspective on these questions and challenges above. It will provide a platform to connect to research groups and researchers across the field (a non-negligible part in the life of a postdoc, of course). And it will spark that extra motivation and excitement needed to tackle the broader, the more important, and the more daunting scientific challenges by bringing together an interdisciplinary, diverse, but in terms of research interests sufficiently cohesive team. 

I would be very happy to be a part of this school and workshop.

With my best wishes, 

Peter Steiglechner

